\documentclass{article}
\title{King + 2 pawns vs. King}
\author{https://lichess.org/@/sasadangelo}
\date{2024.07.09}
\usepackage{svg}
\begin{document}
\begin{titlepage}
\maketitle
\end{titlepage}
\section{ Introduction}
\includesvg[width=150pt]{king-2-pawns-vs-king-endgame/king-2-pawns-vs-king-endgame_1.svg}
\\
\\
We already know what happens with a single pawn. Obviously, a King with two Pawns almost always wins (95 on 100, according to our stats) . This figure would be even higher if we left aside those cases where the capture of one of the pawns is forced. We can distinguish three scenarios:\\\\1) Connected pawns. Here both pawns defend each other and then the king approaches to give them support. The ending is always won unless a pawn is captured. The only important thing to know is this: if we have a rook's pawn on the 7th rank, we have to give it up in order to promote the knight's pawn.\\2) Doubled pawns\\3) Isolated pawns\section{ Doubled Pawns 1}
\includesvg[width=150pt]{king-2-pawns-vs-king-endgame/king-2-pawns-vs-king-endgame_2.svg}
\\
\\
The most interesting case in this section. If we have rook's pawns we know the important thing is not the number of Pawns but the position of the King. Let us see what happens with other pawns.\\\\The standard (and easy) procedure: to use the less advanced pawn to waste a move when the critical position (king opposition on the 6th rank) arises. Here White follows this procedure, but he has to be careful due to two special circumstances worth noting :\\A) We are dealing with knight's pawns, which usually involves stalemate motifs.\\B) The pawns are together, which complicates the defence of the more advanced pawn. The following variation illustrates this idea.\\
\\
\includesvg[width=150pt]{king-2-pawns-vs-king-endgame/king-2-pawns-vs-king-endgame_3.svg}
\\
\\
\textbf{White 1. Kc4}\\
\\
Kc4 is a bad move for White. Black can defend with Kb6.\\
\\
\includesvg[width=150pt]{king-2-pawns-vs-king-endgame/king-2-pawns-vs-king-endgame_4.svg}
\\
\\
\textbf{Black 1... Kb6}\\
\\
At this point, the Pawn in b5 is lost and there is no way White can force a win.\section{ Doubled Pawns 2}
\includesvg[width=150pt]{king-2-pawns-vs-king-endgame/king-2-pawns-vs-king-endgame_5.svg}
\\
\\
The best move for White is Kc3.\\
\\
\includesvg[width=150pt]{king-2-pawns-vs-king-endgame/king-2-pawns-vs-king-endgame_6.svg}
\\
\\
\textbf{White 2. Kc3}\\
\\
Black move Kc7.\\
\\
\includesvg[width=150pt]{king-2-pawns-vs-king-endgame/king-2-pawns-vs-king-endgame_7.svg}
\\
\\
\textbf{Black 2... Kc7}\\
\\
Let's see how this variation continue.\\
\\
\includesvg[width=150pt]{king-2-pawns-vs-king-endgame/king-2-pawns-vs-king-endgame_8.svg}
\\
\\
\textbf{White 3. Kd4}\\
\\
\\
\\
\includesvg[width=150pt]{king-2-pawns-vs-king-endgame/king-2-pawns-vs-king-endgame_9.svg}
\\
\\
\textbf{Black 3... Kb6}\\
\\
\\
\\
\includesvg[width=150pt]{king-2-pawns-vs-king-endgame/king-2-pawns-vs-king-endgame_10.svg}
\\
\\
\textbf{White 4. Kc4}\\
\\
\\
\\
\includesvg[width=150pt]{king-2-pawns-vs-king-endgame/king-2-pawns-vs-king-endgame_11.svg}
\\
\\
\textbf{Black 4... Kc7}\\
\\
\\
\\
\includesvg[width=150pt]{king-2-pawns-vs-king-endgame/king-2-pawns-vs-king-endgame_12.svg}
\\
\\
\textbf{White 5. Kc5}\\
\\
\\
\\
\includesvg[width=150pt]{king-2-pawns-vs-king-endgame/king-2-pawns-vs-king-endgame_13.svg}
\\
\\
\textbf{Black 5... Kb7}\\
\\
\\
\\
\includesvg[width=150pt]{king-2-pawns-vs-king-endgame/king-2-pawns-vs-king-endgame_14.svg}
\\
\\
\textbf{White 6. b6}\\
\\
\\
\\
\includesvg[width=150pt]{king-2-pawns-vs-king-endgame/king-2-pawns-vs-king-endgame_15.svg}
\\
\\
\textbf{Black 6... Ka6}\\
\\
White can't play Kc6 because it is stalemate.\\
\\
\includesvg[width=150pt]{king-2-pawns-vs-king-endgame/king-2-pawns-vs-king-endgame_16.svg}
\\
\\
\textbf{White 7. b7}\\
\\
Pay attention here, Kc6 is stalemate.\\
\\
\includesvg[width=150pt]{king-2-pawns-vs-king-endgame/king-2-pawns-vs-king-endgame_17.svg}
\\
\\
\textbf{Black 7... Kxb7}\\
\\
\\
\\
\includesvg[width=150pt]{king-2-pawns-vs-king-endgame/king-2-pawns-vs-king-endgame_18.svg}
\\
\\
\textbf{White 8. Kb5}\\
\\
White has the opposition and win the game (see King + Pawn vs King study).\section{ Doubled Pawns 3}
\includesvg[width=150pt]{king-2-pawns-vs-king-endgame/king-2-pawns-vs-king-endgame_19.svg}
\\
\\
\includesvg[width=150pt]{king-2-pawns-vs-king-endgame/king-2-pawns-vs-king-endgame_20.svg}
\\
\\
\textbf{White 8. Kc3}\\
\\
\\
\\
\includesvg[width=150pt]{king-2-pawns-vs-king-endgame/king-2-pawns-vs-king-endgame_21.svg}
\\
\\
\textbf{Black 8... Kc7}\\
\\
\\
\\
\includesvg[width=150pt]{king-2-pawns-vs-king-endgame/king-2-pawns-vs-king-endgame_22.svg}
\\
\\
\textbf{White 9. Kd4}\\
\\
\\
\\
\includesvg[width=150pt]{king-2-pawns-vs-king-endgame/king-2-pawns-vs-king-endgame_23.svg}
\\
\\
\textbf{Black 9... Kb6}\\
\\
\\
\\
\includesvg[width=150pt]{king-2-pawns-vs-king-endgame/king-2-pawns-vs-king-endgame_24.svg}
\\
\\
\textbf{White 10. Kc4}\\
\\
\\
\\
\includesvg[width=150pt]{king-2-pawns-vs-king-endgame/king-2-pawns-vs-king-endgame_25.svg}
\\
\\
\textbf{Black 10... Kc7}\\
\\
\\
\\
\includesvg[width=150pt]{king-2-pawns-vs-king-endgame/king-2-pawns-vs-king-endgame_26.svg}
\\
\\
\textbf{White 11. Kc5}\\
\\
\\
\\
\includesvg[width=150pt]{king-2-pawns-vs-king-endgame/king-2-pawns-vs-king-endgame_27.svg}
\\
\\
\textbf{Black 11... Kb7}\\
\\
\\
\\
\includesvg[width=150pt]{king-2-pawns-vs-king-endgame/king-2-pawns-vs-king-endgame_28.svg}
\\
\\
\textbf{White 12. b6}\\
\\
Trying to trouble White as much as possible. Direct play would go as follows:\\
\\
\includesvg[width=150pt]{king-2-pawns-vs-king-endgame/king-2-pawns-vs-king-endgame_29.svg}
\\
\\
\textbf{Black 12... Kb8}\\
\\
\\
\\
\includesvg[width=150pt]{king-2-pawns-vs-king-endgame/king-2-pawns-vs-king-endgame_30.svg}
\\
\\
\textbf{White 13. Kb5}\\
\\
\\
\\
\includesvg[width=150pt]{king-2-pawns-vs-king-endgame/king-2-pawns-vs-king-endgame_31.svg}
\\
\\
\textbf{Black 13... Kc8}\\
\\
\\
\\
\includesvg[width=150pt]{king-2-pawns-vs-king-endgame/king-2-pawns-vs-king-endgame_32.svg}
\\
\\
\textbf{White 14. b7+}\\
\\
\\
\\
\includesvg[width=150pt]{king-2-pawns-vs-king-endgame/king-2-pawns-vs-king-endgame_33.svg}
\\
\\
\textbf{Black 14... Kb8}\\
\\
\\
\\
\includesvg[width=150pt]{king-2-pawns-vs-king-endgame/king-2-pawns-vs-king-endgame_34.svg}
\\
\\
\textbf{White 15. Kc5}\\
\\
White must give up the most advanced pawn to gain the key squares of the other pawn.\\
\\
\includesvg[width=150pt]{king-2-pawns-vs-king-endgame/king-2-pawns-vs-king-endgame_35.svg}
\\
\\
\textbf{Black 15... Kxb7}\\
\\
\\
\\
\includesvg[width=150pt]{king-2-pawns-vs-king-endgame/king-2-pawns-vs-king-endgame_36.svg}
\\
\\
\textbf{White 16. Kb5}\\
\\
White has the opposition and win the game (see King + Pawn vs King study).\section{ Doubled Pawns 4}
\includesvg[width=150pt]{king-2-pawns-vs-king-endgame/king-2-pawns-vs-king-endgame_37.svg}
\\
\\
\includesvg[width=150pt]{king-2-pawns-vs-king-endgame/king-2-pawns-vs-king-endgame_38.svg}
\\
\\
\textbf{White 16. Kc3}\\
\\
\\
\\
\includesvg[width=150pt]{king-2-pawns-vs-king-endgame/king-2-pawns-vs-king-endgame_39.svg}
\\
\\
\textbf{Black 16... Kc7}\\
\\
\\
\\
\includesvg[width=150pt]{king-2-pawns-vs-king-endgame/king-2-pawns-vs-king-endgame_40.svg}
\\
\\
\textbf{White 17. Kd4}\\
\\
\\
\\
\includesvg[width=150pt]{king-2-pawns-vs-king-endgame/king-2-pawns-vs-king-endgame_41.svg}
\\
\\
\textbf{Black 17... Kb6}\\
\\
\\
\\
\includesvg[width=150pt]{king-2-pawns-vs-king-endgame/king-2-pawns-vs-king-endgame_42.svg}
\\
\\
\textbf{White 18. Kc4}\\
\\
\\
\\
\includesvg[width=150pt]{king-2-pawns-vs-king-endgame/king-2-pawns-vs-king-endgame_43.svg}
\\
\\
\textbf{Black 18... Kc7}\\
\\
\\
\\
\includesvg[width=150pt]{king-2-pawns-vs-king-endgame/king-2-pawns-vs-king-endgame_44.svg}
\\
\\
\textbf{White 19. Kc5}\\
\\
\\
\\
\includesvg[width=150pt]{king-2-pawns-vs-king-endgame/king-2-pawns-vs-king-endgame_45.svg}
\\
\\
\textbf{Black 19... Kb7}\\
\\
\\
\\
\includesvg[width=150pt]{king-2-pawns-vs-king-endgame/king-2-pawns-vs-king-endgame_46.svg}
\\
\\
\textbf{White 20. b6}\\
\\
\\
\\
\includesvg[width=150pt]{king-2-pawns-vs-king-endgame/king-2-pawns-vs-king-endgame_47.svg}
\\
\\
\textbf{Black 20... Kb8}\\
\\
\\
\\
\includesvg[width=150pt]{king-2-pawns-vs-king-endgame/king-2-pawns-vs-king-endgame_48.svg}
\\
\\
\textbf{White 21. Kc6}\\
\\
\\
\\
\includesvg[width=150pt]{king-2-pawns-vs-king-endgame/king-2-pawns-vs-king-endgame_49.svg}
\\
\\
\textbf{Black 21... Kc8}\\
\\
\\
\\
\includesvg[width=150pt]{king-2-pawns-vs-king-endgame/king-2-pawns-vs-king-endgame_50.svg}
\\
\\
\textbf{White 22. b7+}\\
\\
\\
\\
\includesvg[width=150pt]{king-2-pawns-vs-king-endgame/king-2-pawns-vs-king-endgame_51.svg}
\\
\\
\textbf{Black 22... Kb8}\\
\\
b5 looks logical\\
\\
\includesvg[width=150pt]{king-2-pawns-vs-king-endgame/king-2-pawns-vs-king-endgame_52.svg}
\\
\\
\textbf{White 23. b5}\\
\\
\\
\\
\includesvg[width=150pt]{king-2-pawns-vs-king-endgame/king-2-pawns-vs-king-endgame_53.svg}
\\
\\
\textbf{Black 23... Ka7}\\
\\
In this position, White has to give up the more advanced pawn, because Kc7 is stalemate.\\
\\
\includesvg[width=150pt]{king-2-pawns-vs-king-endgame/king-2-pawns-vs-king-endgame_54.svg}
\\
\\
\textbf{White 24. b8=Q+}\\
\\
\\
\\
\includesvg[width=150pt]{king-2-pawns-vs-king-endgame/king-2-pawns-vs-king-endgame_55.svg}
\\
\\
\textbf{Black 24... Kxb8}\\
\\
\\
\\
\includesvg[width=150pt]{king-2-pawns-vs-king-endgame/king-2-pawns-vs-king-endgame_56.svg}
\\
\\
\textbf{White 25. Kb6}\\
\\
White has the opposition and win the game (see King + Pawn vs King study). It is easy to see that, if we cannot lose a move with the less advanced pawn, it is impossible to win. This situation arises if the pawn is on the 5th rank. If you have doubts, I recommend that you check it as an Exercise.\\Conclusion: Two doubled pawns always win, except in these two situations:\\1) Rook's pawns.\\2) The less advanced pawn is on the 5th rank.\section{ Isolated Pawns - Introduction}
\includesvg[width=150pt]{king-2-pawns-vs-king-endgame/king-2-pawns-vs-king-endgame_57.svg}
\\
\\
Here the winning chances are great, too. The only situation worth studying occurs when the enemy king threatens the pawns. However, usually the pawns have one of these resources at their disposal:\\\\1) Mutual defence: one pawn threatens to promote to avoid the capture of the other one.\\2) Delay the capture of one pawn until the strong king arrives to defend them and reaches the key squares of the other one.\section{ Isolted Pawns - Mutual defense}
\includesvg[width=150pt]{king-2-pawns-vs-king-endgame/king-2-pawns-vs-king-endgame_58.svg}
\\
\\
This is the first resource: Two pawns separated by one file can defend each other as long as the king does not attack the more advanced one.\\
\\
\includesvg[width=150pt]{king-2-pawns-vs-king-endgame/king-2-pawns-vs-king-endgame_59.svg}
\\
\\
\textbf{White 25. h5}\\
\\
Thus the h-pawn prevents the capture of the f-pawn and gives the white king time to come near. If the king captures the f-pawn, he will be outside the square of the h-pawn. For this reason, he moves Kf6.\\
\\
\includesvg[width=150pt]{king-2-pawns-vs-king-endgame/king-2-pawns-vs-king-endgame_60.svg}
\\
\\
\textbf{Black 25... Kf6}\\
\\
\\
\\
\includesvg[width=150pt]{king-2-pawns-vs-king-endgame/king-2-pawns-vs-king-endgame_61.svg}
\\
\\
\textbf{White 26. Kb2}\\
\\
\\
\\
\includesvg[width=150pt]{king-2-pawns-vs-king-endgame/king-2-pawns-vs-king-endgame_62.svg}
\\
\\
\textbf{Black 26... Kg7}\\
\\
The king is about to capture the advanced pawn, but ...\\
\\
\includesvg[width=150pt]{king-2-pawns-vs-king-endgame/king-2-pawns-vs-king-endgame_63.svg}
\\
\\
\textbf{White 27. f5}\\
\\
\\
\\
\includesvg[width=150pt]{king-2-pawns-vs-king-endgame/king-2-pawns-vs-king-endgame_64.svg}
\\
\\
\textbf{Black 27... Kh6}\\
\\
\\
\\
\includesvg[width=150pt]{king-2-pawns-vs-king-endgame/king-2-pawns-vs-king-endgame_65.svg}
\\
\\
\textbf{White 28. f6}\\
\\
Here we have the same situation as on move 1. The f6 pawn prevents the capture of the h5 pawn. Now White just has to bring the king near and promote one pawn. This is a very useful procedure, especially when there are more pawns on the board. If the pawns are separated by more files, the situation becomes more interesting, assuming, of course, that there are more pawns on the board. Otherwise, it is almost impossible that the stronger side's king does not manage to secure promotion of one pawn.\section{ Isolated Pawns - Delay the Capture}
\includesvg[width=150pt]{king-2-pawns-vs-king-endgame/king-2-pawns-vs-king-endgame_66.svg}
\\
\\
It is Black's turn. However, there is not a single square on the whole board from which the white king fails to support his pawns in time. Let us see a possible sequence of play.\\
\\
\includesvg[width=150pt]{king-2-pawns-vs-king-endgame/king-2-pawns-vs-king-endgame_67.svg}
\\
\\
\textbf{Black 28... Kd5}\\
\\
\\
\\
\includesvg[width=150pt]{king-2-pawns-vs-king-endgame/king-2-pawns-vs-king-endgame_68.svg}
\\
\\
\textbf{White 29. a5}\\
\\
Preventing the capture of the central pawn.\\
\\
\includesvg[width=150pt]{king-2-pawns-vs-king-endgame/king-2-pawns-vs-king-endgame_69.svg}
\\
\\
\textbf{Black 29... Kc6}\\
\\
\\
\\
\includesvg[width=150pt]{king-2-pawns-vs-king-endgame/king-2-pawns-vs-king-endgame_70.svg}
\\
\\
\textbf{White 30. Kg2}\\
\\
\\
\\
\includesvg[width=150pt]{king-2-pawns-vs-king-endgame/king-2-pawns-vs-king-endgame_71.svg}
\\
\\
\textbf{Black 30... Kb5}\\
\\
\\
\\
\includesvg[width=150pt]{king-2-pawns-vs-king-endgame/king-2-pawns-vs-king-endgame_72.svg}
\\
\\
\textbf{White 31. Kf3}\\
\\
\\
\\
\includesvg[width=150pt]{king-2-pawns-vs-king-endgame/king-2-pawns-vs-king-endgame_73.svg}
\\
\\
\textbf{Black 31... Kxa5}\\
\\
\\
\\
\includesvg[width=150pt]{king-2-pawns-vs-king-endgame/king-2-pawns-vs-king-endgame_74.svg}
\\
\\
\textbf{White 32. Ke4}\\
\\
\\
\\
\includesvg[width=150pt]{king-2-pawns-vs-king-endgame/king-2-pawns-vs-king-endgame_75.svg}
\\
\\
\textbf{Black 32... Kb6}\\
\\
\\
\\
\includesvg[width=150pt]{king-2-pawns-vs-king-endgame/king-2-pawns-vs-king-endgame_76.svg}
\\
\\
\textbf{White 33. Kd5}\\
\\
\\
\\
\includesvg[width=150pt]{king-2-pawns-vs-king-endgame/king-2-pawns-vs-king-endgame_77.svg}
\\
\\
\textbf{Black 33... Kc7}\\
\\
\\
\\
\includesvg[width=150pt]{king-2-pawns-vs-king-endgame/king-2-pawns-vs-king-endgame_78.svg}
\\
\\
\textbf{White 34. Ke6}\\
\\
The white king has reached one of the key squares of the d4 pawn. Therefore, the most important scenario arises with more pawns on the board. We will come back to this idea when we deal with the floating square. White won the game. \end{document}
