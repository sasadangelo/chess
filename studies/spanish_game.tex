\documentclass{article}
\title{Spanish Game}
\author{https://lichess.org/@/sasadangelo}
\date{2022.02.09}
\usepackage{svg}
\begin{document}
\begin{titlepage}
\maketitle
\end{titlepage}
\section{ Introduction}
\includesvg[width=150pt]{spanish_game/spanish_game_1.svg}
\\
\\
The Spanish Game is also called Ruy Lopez from the name of the Catholic priest, chess player, and author that made great contributions to chess opening theory with this opening and the King's Gambit. The theory of this opening is very large and it includes a lot of variations classified with the code from 060 to 099 in the Encyclopaedia of Chess Openings (ECO). In this study, I will only talk about a few of them that I consider most important for beginners. From this position, we can have different continuations that we will analyze in the next chapters.\section{ Spanish Game}
\includesvg[width=150pt]{spanish_game/spanish_game_2.svg}
\\
\\
Spanish Game starts with the King's Opening: Two Knights Variation (see Basic Opening and Principles study). In order to play the Italian Game, you need to move Bb5.\\\\
\\
\includesvg[width=150pt]{spanish_game/spanish_game_3.svg}
\\
\\
\textbf{White 1. Bb5}\\
\\
This configuration on the Chessboard is the Spanish Game. White's goals with this opening are many:\\1) Rapid development.\\2) Castle quickly.\\3) Put pressure on the Black Knight on c6 on the diagonal leading to the King. More generally, White will have the possibility of pressure for a long time.\\4) Good lines for both tactical and positional players.\\\section{ Morphy Defense}
\includesvg[width=150pt]{spanish_game/spanish_game_4.svg}
\\
\\
The classical Black reply to the Spanish Game is a6 to press the White Bishop and force it to capture the Black Knight. This move is called Morphy Defense.\\
\\
\includesvg[width=150pt]{spanish_game/spanish_game_5.svg}
\\
\\
\textbf{Black 1... a6}\\
\\
The goal of the Morphy Defense is to pressure the White Bishop on b5 and force him to capture the Black Knight. Unfortunately, if White takes the bait, unpleasant scenarios will open up for him.\\The Morphy Defense has three main lines:\\1) Main-line;\\2) Closed Defense;\\3) Other Alternatives.\\\section{ Morphy Defense - Exchange Variation}
\includesvg[width=150pt]{spanish_game/spanish_game_6.svg}
\\
\\
The Exchange Variation in the Morphy Defense is one of the unpleasant scenarios for White and it is one of the alternative variations to the mainline and the closed variation. The reason why I report here this variation is only to show why it is a bad idea for White to take the bait and capture the Black Knight. Move Bxc6.\\
\\
\includesvg[width=150pt]{spanish_game/spanish_game_7.svg}
\\
\\
\textbf{White 3. Bxc6}\\
\\
White Bishop takes the bait and moves Bxc6 capturing the Black Knight.\\
\\
\includesvg[width=150pt]{spanish_game/spanish_game_8.svg}
\\
\\
\textbf{Black 3... dxc6}\\
\\
Black replies with dxc6 ruining his Pawn structure in exchange for a Bishop. Black could have made the move bxc6 which, however, would have led to moves d4, exd4, and Qxd4 which gave White control of the center. Instead, the move dxc6 opens the diagonal c8-h3 to the bishop and the d-column to the queen. White move Nxe5.\\
\\
\includesvg[width=150pt]{spanish_game/spanish_game_9.svg}
\\
\\
\textbf{White 4. Nxe5}\\
\\
At this point, the White Knight captures the pawn on e5 with the move Nxe5 and, at the same time, threatens the pawn on c6.\\\\
\\
\includesvg[width=150pt]{spanish_game/spanish_game_10.svg}
\\
\\
\textbf{Black 4... Qd4}\\
\\
The Black Queen will reply with Qd4 by attacking both the White Pawn on e4 and the White Knight on e5 in the center. In addition, it will also put pressure on the d2 pawn on the White Queen line. Not a favorable line for White that is force to move Nf3.\\\\
\\
\includesvg[width=150pt]{spanish_game/spanish_game_11.svg}
\\
\\
\textbf{White 5. Nf3}\\
\\
The White Knight is forced to move Nc3 attacking the Queen.\\\\
\\
\includesvg[width=150pt]{spanish_game/spanish_game_12.svg}
\\
\\
\textbf{Black 5... Qxe4+}\\
\\
The Black Queen will capture the pawn with Qxe4+ controlling the center threatening the White Knight and checking the King. It's clear then that the Exchange Variation is not a good alternative for White.\\\section{ Morphy Defense - Columbus and Caro Variations}
\includesvg[width=150pt]{spanish_game/spanish_game_13.svg}
\\
\\
In this chapter, we will analyze how a beginner should reply to the Morphy Defense move a6. The best move for White is Ba4.\\
\\
\includesvg[width=150pt]{spanish_game/spanish_game_14.svg}
\\
\\
\textbf{White 4. Ba4}\\
\\
The goal of this move is to remove the a6 Pawn threat and continue to put pressure on the Black Knight on the diagonal leading to the King. This variation is called Columbus.\\\\
\\
\includesvg[width=150pt]{spanish_game/spanish_game_15.svg}
\\
\\
\textbf{Black 4... b5}\\
\\
Black can reply with b5 (Caro variation) forcing the White Bishop to move Bb3 with a good diagonal on the weak pawn f7.\\\\
\\
\includesvg[width=150pt]{spanish_game/spanish_game_16.svg}
\\
\\
\textbf{White 5. Bb3}\\
\\
White moving Bb3 remove the b5 Black Pawn threat making pressure on weak f7 Black Pawn.\\\section{ Morphy Defense - Closed Variation}
\includesvg[width=150pt]{spanish_game/spanish_game_17.svg}
\\
\\
The Closed Variation is a continuation of the Columbus Variation where instead of b5 Black play Nf6.\\
\\
\includesvg[width=150pt]{spanish_game/spanish_game_18.svg}
\\
\\
\textbf{Black 5... Nf6}\\
\\
Black play Nf6 to threaten White Pawn in e4 and defend the d5 square. White protects the King with a castle.\\\\
\\
\includesvg[width=150pt]{spanish_game/spanish_game_19.svg}
\\
\\
\textbf{White 6. O-O}\\
\\
White King is protected and Black prepares for a similar move.\\
\\
\includesvg[width=150pt]{spanish_game/spanish_game_20.svg}
\\
\\
\textbf{Black 6... Be7}\\
\\
Black move Bb7 to prepare its King to castle.\section{ Fide World Championship 2021}
\includesvg[width=150pt]{spanish_game/spanish_game_21.svg}
\\
\\
In the Fide World Chess Championship of 2021 final match between the champion Magnus Carlsen (White) and the winner of the tournament of the candidates, Nepomniachtchi (Black) in some matches (Games 1, 3, and 5) played the Closed Variation of the Spanish Game. In this chapter, we analyze how they continued the opening before entering the middle game. White move Re1.\\
\\
\includesvg[width=150pt]{spanish_game/spanish_game_22.svg}
\\
\\
\textbf{White 8. Re1}\\
\\
White moves Re1 to have a more active role on the e column and defend the White Pawn in e4.\\\\
\\
\includesvg[width=150pt]{spanish_game/spanish_game_23.svg}
\\
\\
\textbf{Black 8... b5}\\
\\
Black replied with the b5 move we saw in the Caro variation. White should protect the Bishop with Bb3.\\\\
\\
\includesvg[width=150pt]{spanish_game/spanish_game_24.svg}
\\
\\
\textbf{White 9. Bb3}\\
\\
White is thus forced to move Bb3 by opening the diagonal on the weak pawn on f7.\\\\
\\
\includesvg[width=150pt]{spanish_game/spanish_game_25.svg}
\\
\\
\textbf{Black 9... O-O}\\
\\
Black replies with O-O castling.\section{ Marshall Attack}
\includesvg[width=150pt]{spanish_game/spanish_game_26.svg}
\\
\\
Starting from the final position in the Magnus Carlsen (White) vs Nepomniachtchi (Black) game studied in the previous chapter, we will study here the Marshall Attack a continuation for Black if White play c3. Play c3.\\
\\
\includesvg[width=150pt]{spanish_game/spanish_game_27.svg}
\\
\\
\textbf{White 10. c3}\\
\\
White continues its development playing c3 to control the center hoping to move d4 in the next move.\\
\\
\includesvg[width=150pt]{spanish_game/spanish_game_28.svg}
\\
\\
\textbf{Black 10... d5}\\
\\
Black replies with d5. This configuration on the chessboard is called Marshall Attack.\\The Marshall Attack was introduced by Frank Marshall in a famous game against Capablanca in 1918. According to legend, Marshall saved this prepared innovation for eight years before getting the chance to play it against Capablanca. This seems unlikely and, in fact, the gambit had been played earlier in a few obscure games. Capablanca weathered the Black attack and won brilliantly. Improvements to Black's play were found and the Marshall Attack was adopted by top players.\\The goal for Black is to sacrifice a Pawn to gain better development in the next moves. Let's see now a possible continuation of this attack.\\
\\
\includesvg[width=150pt]{spanish_game/spanish_game_29.svg}
\\
\\
\textbf{White 11. exd5}\\
\\
White capture the Black Pawn with exd5.\\
\\
\includesvg[width=150pt]{spanish_game/spanish_game_30.svg}
\\
\\
\textbf{Black 11... Nxd5}\\
\\
Black replies with Nxd5.\\
\\
\includesvg[width=150pt]{spanish_game/spanish_game_31.svg}
\\
\\
\textbf{White 12. Nxe5}\\
\\
White moves Nxe5.\\
\\
\includesvg[width=150pt]{spanish_game/spanish_game_32.svg}
\\
\\
\textbf{Black 12... Nxe5}\\
\\
Black captures the White Knight with Nxe5.\\
\\
\includesvg[width=150pt]{spanish_game/spanish_game_33.svg}
\\
\\
\textbf{White 13. Rxe5}\\
\\
White captures the Knight with Rxe5. The problem now for Black is that it has a Knight unprotected in d5. For this reason, it moves c4.\\
\\
\includesvg[width=150pt]{spanish_game/spanish_game_34.svg}
\\
\\
\textbf{Black 13... c6}\\
\\
White has the same problem with an unprotected Rook in e5. For this reason, it moves d4.\\
\\
\includesvg[width=150pt]{spanish_game/spanish_game_35.svg}
\\
\\
\textbf{White 14. d4}\\
\\
Black threatens the White Rook with the bishop Bd6.\\
\\
\includesvg[width=150pt]{spanish_game/spanish_game_36.svg}
\\
\\
\textbf{Black 14... Bd6}\\
\\
White is forced to move the Rook back to e1.\\
\\
\includesvg[width=150pt]{spanish_game/spanish_game_37.svg}
\\
\\
\textbf{White 15. Re1}\\
\\
Black is now ready to launch his attack with the Queen on h4 with the move Qh4. With this move Black threatens checkmate with the Queen on h2 on the next move.\\
\\
\includesvg[width=150pt]{spanish_game/spanish_game_38.svg}
\\
\\
\textbf{Black 15... Qh4}\\
\\
The only way for White to defend himself from checkmate is g3.\\
\\
\includesvg[width=150pt]{spanish_game/spanish_game_39.svg}
\\
\\
\textbf{White 16. g3}\\
\\
Black replies with Qh3.\\
\\
\includesvg[width=150pt]{spanish_game/spanish_game_40.svg}
\\
\\
\textbf{Black 16... Qh3}\\
\\
There are some possible continuations starting from this position but for the moment I think that what we have analyzed is enough.\end{document}
