\documentclass{article}
\title{Spanish Game}
\author{https://lichess.org/@/sasadangelo}
\date{2022.02.03}
\usepackage{svg}
\begin{document}
\begin{titlepage}
\maketitle
\end{titlepage}
\section{ Introduction}

The Spanish Game is also called Ruy Lopez from the name of the Catholic priest, chess player, and author that made great contributions to chess opening theory with this opening and the King's Gambit. The theory of this opening is very large and it includes a lot of variations classified with the code from 060 to 099 in the Encyclopaedia of Chess Openings (ECO). In this study, I will only talk about a few of them that I consider most important for beginners. From this position, we can have different continuations that we will analyze in the next chapters.\\
\\
\section{ Spanish Game}

Spanish Game (or Ruy Lopez) starts with the King’s Opening e4. Play e4, you will open the diagonals for White Queen and Bishop.\\
\\
\includesvg[width=150pt]{spanish-game/spanish-game_1.svg}
\\
\\
\textbf{White 1. e4}\\
\\
Black replies with e5 with the same motivation.\\
\\
\includesvg[width=150pt]{spanish-game/spanish-game_2.svg}
\\
\\
\textbf{Black 1... e5}\\
\\
This configuration on the Chessboard is called King's Opening. Move Nf3 in order to attack the Pawn in e5.\\
\\
\includesvg[width=150pt]{spanish-game/spanish-game_3.svg}
\\
\\
\textbf{White 2. Nf3}\\
\\
Black replies with Nc6 in order to defend the Pawn in e5.\\
\\
\includesvg[width=150pt]{spanish-game/spanish-game_4.svg}
\\
\\
\textbf{Black 2... Nc6}\\
\\
This line on the Chessboard is called King's Opening: Knights Variation. In order to play the Spanish Game, you need to move Bb5.\\
\\
\includesvg[width=150pt]{spanish-game/spanish-game_5.svg}
\\
\\
\textbf{White 3. Bb5}\\
\\
This configuration on the Chessboard is the Spanish Game. White's goals with this opening are many:\\1) Rapid development.\\2) Castle quickly.\\3) Put pressure on the Black Knight on c6 on the diagonal leading to the King. More generally, White will have the possibility of pressure for a long time.\\4) Good lines for both tactical and positional players.\\
\\
\section{ Morphy Defense}

The classical Black reply to the Spanish Game is a6 to press the White Bishop and force it to capture the Black Knight. This move is called Morphy Defense.\\
\\
\includesvg[width=150pt]{spanish-game/spanish-game_6.svg}
\\
\\
\textbf{Black 1... a6}\\
\\
The goal of the Morphy Defense is to pressure the White Bishop on b5 and force him to capture the Black Knight. Unfortunately, if White takes the bait, unpleasant scenarios will open up for him.\\The Morphy Defense has three main lines:\\1) Main-line;\\2) Closed Defense;\\3) Other Alternatives.\\
\\
\section{ Morphy Defense - Exchange Variation}

The Exchange Variation in the Morphy Defense is one of the unpleasant scenarios for White and it is one of the alternative variations to the mainline and the closed variation. The reason why I report here this variation is only to show why it is a bad idea for White to take the bait and capture the Black Knight.\\
\\
\includesvg[width=150pt]{spanish-game/spanish-game_7.svg}
\\
\\
\textbf{White 3. Bxc6}\\
\\
White Bishop takes the bait and moves Bxc6 capturing the Black Knight.\\
\\
\includesvg[width=150pt]{spanish-game/spanish-game_8.svg}
\\
\\
\textbf{Black 3... dxc6}\\
\\
Black replies with dxc6 ruining his Pawn structure in exchange for a Bishop. Black could have made the move bxc6 which, however, would have led to moves d4, exd4, and Qxd4 which gave White control of the center. Instead, the move dxc6 opens the diagonal c8-h3 to the bishop and the d-column to the queen.\\
\\
\includesvg[width=150pt]{spanish-game/spanish-game_9.svg}
\\
\\
\textbf{White 4. Nxe5}\\
\\
At this point, the White Knight captures the pawn on e5 with the move Nxe5 and, at the same time, threatens the pawn on c6.\\
\\
\includesvg[width=150pt]{spanish-game/spanish-game_10.svg}
\\
\\
\textbf{Black 4... Qd4}\\
\\
The Black Queen will reply with Qd4 by attacking both the White Pawn on e4 and the White Knight on e5 in the center. In addition, it will also put pressure on the d2 pawn on the White Queen line. Not a favorable line for White.\\
\\
\includesvg[width=150pt]{spanish-game/spanish-game_11.svg}
\\
\\
\textbf{White 5. Nf3}\\
\\
The White Knight is forced to move Nc3.\\
\\
\includesvg[width=150pt]{spanish-game/spanish-game_12.svg}
\\
\\
\textbf{Black 5... Qxe4+}\\
\\
The Black Queen will capture the pawn with Qxe4+ controlling the center threatening the White Knight and checking the King. It's clear then that the Exchange Variation is not a good alternative for White.\\
\\
\section{ Morphy Defense - Columbus and Caro Variations}

In this chapter, we will analyze how a beginner should reply to the Morphy Defense move a6.\\
\\
\includesvg[width=150pt]{spanish-game/spanish-game_13.svg}
\\
\\
\textbf{White 4. Ba4}\\
\\
The best reply for White to the Morphy Defense move a6 is Ba4. The goal is to remove the a6 Pawn threat and continue to put pressure on the Black Knight on the diagonal leading to the King. This variation is called Columbus.\\
\\
\includesvg[width=150pt]{spanish-game/spanish-game_14.svg}
\\
\\
\textbf{Black 4... b5}\\
\\
Black can reply with b5 (Caro variation) forcing the White Bishop to move Bb3 with a good diagonal on the weak pawn f7.\\
\\
\includesvg[width=150pt]{spanish-game/spanish-game_15.svg}
\\
\\
\textbf{White 5. Bb3}\\
\\
White move Bb3 remove the b5 Black Pawn threat and making pressure on weak f7 Black Pawn.\\
\\
\section{ Morphy Defense - Closed Variation}

The Closed Variation is a continuation of the Columbus Variation where instead of b5 Black play Nf6.\\
\\
\includesvg[width=150pt]{spanish-game/spanish-game_16.svg}
\\
\\
\textbf{Black 5... Nf6}\\
\\
Black play Nf6 to threaten White Pawn in e4 and defend the d5 square.\\
\\
\includesvg[width=150pt]{spanish-game/spanish-game_17.svg}
\\
\\
\textbf{White 6. O-O}\\
\\
White protects the King with a castle.\\
\\
\includesvg[width=150pt]{spanish-game/spanish-game_18.svg}
\\
\\
\textbf{Black 6... Be7}\\
\\
Black move Bb7 to prepare its King to castle.\\
\\
\section{ Fide World Championship 2021}

In the Fide World Chess Championship of 2021 final match between the champion Magnus Carlsen (White) and the winner of the tournament of the candidates, Nepomniachtchi (Black) in some matches (Games 1, 3, and 5) played the Closed Variation of the Spanish Game. In this chapter, we analyze how they continued the opening before entering the middle game.\\
\\
\includesvg[width=150pt]{spanish-game/spanish-game_19.svg}
\\
\\
\textbf{White 8. Re1}\\
\\
White moves Re1 to have a more active role on the e column and defend the White Pawn in e4.\\
\\
\includesvg[width=150pt]{spanish-game/spanish-game_20.svg}
\\
\\
\textbf{Black 8... b5}\\
\\
Black replied with the b5 move we saw in the Caro variation.\\
\\
\includesvg[width=150pt]{spanish-game/spanish-game_21.svg}
\\
\\
\textbf{White 9. Bb3}\\
\\
White is thus forced to move Bb3 by opening the diagonal on the weak pawn on f7.\\
\\
\includesvg[width=150pt]{spanish-game/spanish-game_22.svg}
\\
\\
\textbf{Black 9... O-O}\\
\\
Black replies with O-O castling.\\
\\
\end{document}
