\documentclass{article}
\title{Wayward Queen Attack}
\author{https://lichess.org/@/sasadangelo}
\date{2022.09.27}
\usepackage{svg}
\begin{document}
\begin{titlepage}
\maketitle
\end{titlepage}
\section{ Introduction}
\includesvg[width=150pt]{wayward-queen-attack/wayward-queen-attack_1.svg}
\\
\\
During my games to reach the score of 500 played with Black, I have often come across users playing the Wayward attack opening with the queen. This opening was considered by many experienced users to be unsuitable for a satisfying game because, if well defended, it inevitably leads to a disadvantage for white at the end of the development of the opening. The problem, however, is that when playing with very low level players (eg 0-500) it is easy for those who play with black do not know how to defend themselves and in the end White manages to win by checkmate in a few moves.\\\\Here I report 3 games in which I suffered this attack without knowing how to respond:\\https://www.chess.com/game/live/30494370735\\https://www.chess.com/game/live/30305952817\\https://www.chess.com/game/live/29508579067\\\\The goal of this article is to understand how to defend against this attack.\\\\Wayward's opening includes the following moves: 1. e4, e5; 2. Rh5, ...; thus arriving at a chessboard configuration like the diagraam.\\\\As you can see, the Queen on h5 threatens the pawn on e5 and the weak pawn on f7. Here the player has two defense possibilities.\section{ Main Line}
\includesvg[width=150pt]{wayward-queen-attack/wayward-queen-attack_2.svg}
\\
\\
We defend the pawn on e5 with Nc6. In this way the Queen will not be able to eat the pawn and put the King in check. As can be seen from the following figure, however, the threat of the Queen remains on the pawn f7. However, it is not convenient for White to mistake a pawn for a queen.\\\\
\\
\includesvg[width=150pt]{wayward-queen-attack/wayward-queen-attack_3.svg}
\\
\\
\textbf{Black 1... Nc6}\\
\\
At this point, suppose White plays Bc4. The reason for this move is to increase the pressure on the weak pawn f7.\\\\
\\
\includesvg[width=150pt]{wayward-queen-attack/wayward-queen-attack_4.svg}
\\
\\
\textbf{White 2. Bc4}\\
\\
White responds with g6 protecting the pawn on f7 from the Queen.\\\\
\\
\includesvg[width=150pt]{wayward-queen-attack/wayward-queen-attack_5.svg}
\\
\\
\textbf{Black 2... g6}\\
\\
At this point, only the Bishop will exert pressure on the pawn on f7. The Queen at this point can not help but step back and so the White will waste precious time protecting the Queen rather than developing his game. The most reasonable move that White can make is Qf3 thus renewing the threat on f7 together with the Bishop on c4.\\\\
\\
\includesvg[width=150pt]{wayward-queen-attack/wayward-queen-attack_6.svg}
\\
\\
\textbf{White 3. Qf3}\\
\\
However, Black will defend himself with Nf6 thus putting an end to the threat of the Queen.\\\\
\\
\includesvg[width=150pt]{wayward-queen-attack/wayward-queen-attack_7.svg}
\\
\\
\textbf{Black 3... Nf6}\\
\\
As you can see, the only active threat is only that of Bishop on c4 on pawn f7.\\\\
\\
\includesvg[width=150pt]{wayward-queen-attack/wayward-queen-attack_8.svg}
\\
\\
\textbf{White 4. Qb3}\\
\\
At this point, if the Queen insists with Qb3 bringing a double attack to the pawn on f7, Black defends the pawn with Qe7 and if White moves Bxf7+ the Queens are exchanges and Black gain a Bishop. Another possible line that bring to the same result is Nd4.\\\\
\\
\includesvg[width=150pt]{wayward-queen-attack/wayward-queen-attack_9.svg}
\\
\\
\textbf{Black 4... Nd4}\\
\\
If Bishop decides to capture the f7 pawn checking the King with Bxf7+, this replies with Ke7.\\\\
\\
\includesvg[width=150pt]{wayward-queen-attack/wayward-queen-attack_10.svg}
\\
\\
\textbf{White 5. Bxf7+}\\
\\
Black replies with Ke7.\\
\\
\includesvg[width=150pt]{wayward-queen-attack/wayward-queen-attack_11.svg}
\\
\\
\textbf{Black 5... Ke7}\\
\\
The only move for Queen is Qd3. Even if the Queen move Qc4 we can move b5 and force the Queen on d3 square.\\
\\
\includesvg[width=150pt]{wayward-queen-attack/wayward-queen-attack_12.svg}
\\
\\
\textbf{White 6. Qd3}\\
\\
Black King can now capture the White Bishop.\\
\\
\includesvg[width=150pt]{wayward-queen-attack/wayward-queen-attack_13.svg}
\\
\\
\textbf{Black 6... Kxf7}\\
\\
Congratulations! You have now a better position than White.\section{ Kiddie Countergaambit}
\includesvg[width=150pt]{wayward-queen-attack/wayward-queen-attack_14.svg}
\\
\\
Let's see another variation of the Wayward Queen Attack. Let's go back to the moves: e4, e5, Qh5. This time, however, instead of defending the Pawn on e5 with Nc6.\\\\
\\
\includesvg[width=150pt]{wayward-queen-attack/wayward-queen-attack_15.svg}
\\
\\
\textbf{Black 1... Nf6}\\
\\
Notice how this move does not remove the threats on the e5 and f7 pawns. Let the Queen take the pawn on e5 in such a way, however, that we can continue our development.\\\\
\\
\includesvg[width=150pt]{wayward-queen-attack/wayward-queen-attack_16.svg}
\\
\\
\textbf{White 2. Qxe5+}\\
\\
White then moves with Qxe5 + and checks the King. Black defends the check with Be7.\\
\\
\includesvg[width=150pt]{wayward-queen-attack/wayward-queen-attack_17.svg}
\\
\\
\textbf{Black 2... Be7}\\
\\
As you can see, Black has no compelling threats. At this point any development could continue with Nf3, O-O.\\
\\
\includesvg[width=150pt]{wayward-queen-attack/wayward-queen-attack_18.svg}
\\
\\
\textbf{White 3. Nf3}\\
\\
Black replies with O-O.\\
\\
\includesvg[width=150pt]{wayward-queen-attack/wayward-queen-attack_19.svg}
\\
\\
\textbf{Black 3... O-O}\\
\\
The development could follow with Bc4 putting pressure on the f7 Pawn.\\
\\
\includesvg[width=150pt]{wayward-queen-attack/wayward-queen-attack_20.svg}
\\
\\
\textbf{White 4. Bc4}\\
\\
Black responds with Re8 so as to increase pressure on the Queen.\\
\\
\includesvg[width=150pt]{wayward-queen-attack/wayward-queen-attack_21.svg}
\\
\\
\textbf{Black 4... Re8}\\
\\
If White decides to defend himself with O-O castling, Black can continue with Bb4 leaving the Rook unguarded which, however, is defended by the Queen.\\
\\
\includesvg[width=150pt]{wayward-queen-attack/wayward-queen-attack_22.svg}
\\
\\
\textbf{White 5. O-O}\\
\\
Black continue with Bb4.\\
\\
\includesvg[width=150pt]{wayward-queen-attack/wayward-queen-attack_23.svg}
\\
\\
\textbf{Black 5... Bb4}\\
\\
If White plays d3.\\
\\
\includesvg[width=150pt]{wayward-queen-attack/wayward-queen-attack_24.svg}
\\
\\
\textbf{White 6. d3}\\
\\
Black can respond with d5 square protected by the knight on f6.\\
\\
\includesvg[width=150pt]{wayward-queen-attack/wayward-queen-attack_25.svg}
\\
\\
\textbf{Black 6... d5}\\
\\
Then Blaack can continue with Nxe4.\\\end{document}
